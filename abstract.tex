\documentclass{article}

\begin{document}
	The massive expansion of scientific literature on climate change challenges the Intergovenmental Panel on Climate Change (IPCC)'s ability to assess the science according to its objectives.
	Moreover, the number and variety of papers hinders researchers of the science-policy interface from making objective judgements about those IPCC assessments. In this paper, we present a novel application of a machine-reading approach to model the topical content of 400,000 papers on climate change. This topic model provides the basis for a \textit{topography} of climate change literature. By comparing the climate change literature as a whole to that which is cited by IPCC reports, we show that, contrary to previous estimations \cite{Bjurström2011}, social science research on climate change is relatively well represented by IPCC reports. Conversely, topics on technical solutions to climate change, and knowledge from Engineering and the Agricultural Sciences in general, are under-represented.
\end{document}