\documentclass{letter}
\usepackage{letterbib}
\usepackage{natbib}
\usepackage{geometry}

\usepackage{xcolor}
\usepackage{graphicx}
\usepackage{tikz}
\usepackage{url}
\usepackage{xifthen}


\usepackage{tikzpagenodes}
\geometry{
	a4paper,
	left=20mm,
	right=60mm,
	marginparwidth=50mm,
	top=30mm,
}




\def\myyshifttop{0}
\def\mypagetop{0}

\newcommand{\marpartop}[1]% content, color
{   \begin{tikzpicture}[remember picture, overlay]
	\ifthenelse{\thepage=\mypagetop}{}{\xdef\myyshifttop{0}}        
	\xdef\mypagetop{\thepage}
	\node[below right, yshift=\myyshifttop, text width=\marginparwidth-4pt, inner sep=2pt] (tempnode) at (current page marginpar area.north west) {#1};
	\path (current page marginpar area.north west);
	\pgfgetlastxy{\tempxone}{\tempyone}
	\path (tempnode.south west);
	\pgfgetlastxy{\tempxtwo}{\tempytwo}
	\pgfmathsetmacro{\diffy}{(\tempytwo-\tempyone)}
	\xdef\myyshifttop{\diffy}
	\end{tikzpicture}
}

\def\myyshiftbot{0}
\def\mypagebot{0}

\newcommand{\marparbot}[1]% content, color
{   \begin{tikzpicture}[remember picture, overlay]
	\ifthenelse{\thepage=\mypagebot}{}{\xdef\myyshiftbot{0}}        
	\xdef\mypagebot{\thepage}
	\node[above right, yshift=\myyshiftbot, text width=\marginparwidth-4pt, inner sep=2pt] (tempnode) at (current page marginpar area.south west) {#1};
	\path (current page marginpar area.south west);
	\pgfgetlastxy{\tempxone}{\tempyone}
	\path (tempnode.north west);
	\pgfgetlastxy{\tempxtwo}{\tempytwo}
	\pgfmathsetmacro{\diffy}{(\tempytwo-\tempyone)}
	\xdef\myyshiftbot{\diffy}
	\end{tikzpicture}
}




\signature{Max Callaghan}



\begin{document}
		
	\begin{letter}{Bronwyn Wake \\ Chief Editor \\ \textit{Nature Climate Change}}
		
		
		
		\begin{tikzpicture}[remember picture, overlay]
		\node [anchor=north west, shift={(-1mm,0mm)}]  at (current page text area.north west) {\footnotesize\textbf{MCC} Torgauer Str. 12-15 \textbar\ 10829 Berlin \textbar\  Germany
		};
		\end{tikzpicture}
		
		\begin{tikzpicture}[remember picture, overlay]
		\node [anchor=south, shift={(-1mm,5mm)}]  at (current page text area.north east) {\includegraphics[width=4cm]{pres/MCC_Logo_RZ_rgb.jpg}
		};
		\end{tikzpicture}
		
		\marpartop{\textcolor[HTML]{0067a1}{\footnotesize\textsf{Mercator Research Institute on Global Commons and Climate Change (MCC) gemeinnützige GmbH}\\ \bigskip}%
			\par\textsf{\textbf{Max W. Callaghan}}%
			\par\textsf{PhD Candidate}%
			\par\textsf{Applied Sustainability Science}%
			\par\textsf{Torgauer Str. 12-15}
			\par\textsf{10829 Berlin | Germany}
			\par\textsf{tel +49 (0) 30 338 55 27 - 245}
			\par\textsf{fax +49 (0) 30 338 55 37 - 102}
			\par\textsf{mail callaghan@mcc-berlin.net}
			\par\textsf{web \url{www.mcc-berlin.net}}
		}


		
	
		\marparbot{
			\footnotesize
			Director \\
			Prof. Dr. Ottmar Edenhofer \\
			\medskip
			Bank Account \\
			Sparkasse Berlin \\
			iban DE58 1005 0000 0190 1246 36 \\
			bic  BELADEBEXXX \\
			\medskip
			Amtsgericht Charlottenburg \\
			HRB 142902 B \\
			\medskip
			MCC was founded jointly by the Stiftung Mercator and the Potsdam Institute for Climate Impact Research
		}
	
		\date{}
		
		


   \opening{\today \\ \\ Dear Bronwyn Wake,}

   I herewith submit on behalf of my co-authors the manuscript ``A Topography of Climate Change Litereature'' for consideration as a letter in Nature Climate Change.

``Big literature'' - the vast and rapidly expanding body of publications - poses new, and largely neglected challenges for climate change assessments and how we understand them. For example, we estimate a total of 350,000 new publications on climate change to emerge (only) in the Web of Science during the Sixth Assessment Cycle of the Intergovernmental Panel on Climate Change spanning from 2015 to 2022. Big data and machine learning methods need to be applied to maintain an overview of developments in the scientific landscape and to analyse how past IPCC assessments have engaged with the available literature. So far, attempts \cite{Bjurström2011, Hulme2010, Victor2015, Kowarsch2017} have often been based on qualitative perceptions, or incomplete data with insufficient methodological sophistication.

In this study we use machine-learning to provide a comprehensive, thematic map of the more than 400,000 articles on climate change published in the Web of Science to date. We perceive two innovations that allow us advance the current state of knowledge on the IPCC and the field it is tasked with assessing. First, our thematic map allows us to situate a complete set of IPCC citations into the context of the wider literature and analyse disciplinary representation adequately.  We find that contrary to previous claims, the social sciences are over-represented in IPCC reports.
Second, the use of machine learning methods in creating the map allows us to analyse the content of publications more or less represented in the IPCC. We find that literature on technical solutions (such as negative emissions, buildings or cities) is growing fast and, even compared to other fast-growing literatures, has been under-represented in IPCC reports. We point towards a variety of other applications of such a comprehensive, thematic literature map within global environmetal assessments as well as the broad field of research synthesis. Our findings have direct implications for addressing growing demands for more solution-oriented climate change assessments that are also more firmly rooted in the social sciences.

\closing{Yours Sincerely,}


\bibliography{Mendeley}
\bibliographystyle{unsrt}

\end{letter}
\end{document}