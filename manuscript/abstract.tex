\documentclass{article}

\begin{document}
	The massive expansion of scientific literature on climate change \cite{Minx2017l} poses challenges for global environmental assessments and our understanding of how these assessments work. 
	Big data and machine learning can help us deal with large collections of scientific text, making the production of assessments
	more tractable, and giving us better insights about how past assessments  have engaged with the literature.
	We use topic modelling to draw a topic map, or topography, of over 400,000 publications from the Web of Science (WoS) on climate change. 
	We update current knowledge on the Intergovernmental Panel on Climate Change (IPCC), showing that, when compared to the baseline of the literature identified,  the social sciences are in fact over-represented in recent assessment reports. Technical, solutions-relevant knowledge - especially in agriculture and engineering - is under-represented.
	We suggest a variety of other applications of such maps, and our findings have direct implications for addressing growing demands for more solution-oriented climate change assessments that are also more firmly rooted in the social sciences \cite{Kowarsch2017, Victor2015}.
	The perceived lack of social science knowledge in assessment reports does not necessarily imply a IPCC bias, but rather suggests a need for more social science research with a focus on ``technical'' topics on climate solutions. 
\end{document}