\documentclass{article}

\begin{document}
	The massive expansion of scientific literature on climate change poses challenges for global environmental assessments and our understanding of how these assessments work. 
	Big data and machine learning can help us deal 
	with the large collections of text represented by scientific fields.
	Such methods help make the production of assessments
	more tractable, and give us better insights about how past assessments  have engaged with the literature as it has evolved.
	We use topic modelling to identify the thematic structure and draw a comprehensive topic map, or topography, of over 400,000 scientific publications from the Web of Science (WoS) on climate change. 
	We update current knowledge on the Intergovernmental Panel on Climate Change (IPCC), showing that, at least when compared to the baseline of the literature identified in the WoS,  the social sciences are in fact over-represented in recent assessment reports, and that
	technical, solutions-relevant knowledge - especially in the agricultural and engineering sciences - are under-represented.
	We point to a variety of other applications of such maps, and our findings have direct implications for addressing growing demands for more solution-oriented climate change assessments that are also more firmly rooted in the social sciences.
	We highlight fast-growing topics on solutions that could be better integrated into future IPCC reports. 
	The perceived lack of social science knowledge in solutions-relevant IPCC reports does not necessarily imply a bias towards the natural sciences. 
	It rather suggests a need for more social science research with a focus on ``technical'' topics related to climate solutions. 
\end{document}